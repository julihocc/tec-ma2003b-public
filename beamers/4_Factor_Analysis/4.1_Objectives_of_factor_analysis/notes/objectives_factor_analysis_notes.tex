% Clean, self-contained Tufte-style notes for Topic 4.1: Objectives of Factor Analysis
\documentclass[a4paper]{tufte-book}
\usepackage[utf8]{inputenc}
\usepackage{amsmath,amssymb,amsfonts}
\usepackage{graphicx}
\usepackage{hyperref}
\usepackage{enumitem}
\usepackage{xcolor}
\usepackage{framed}
\usepackage{listings}
\usepackage{tcolorbox}
\usepackage{titlesec}
% Note: natbib is already loaded by tufte-book class

% Tufte classes manage headers/footers; use title/author and tufte sidenotes

% MA2003B branding colors from the beamer theme
\definecolor{tecblue}{RGB}{0,88,170}
\definecolor{tecorange}{RGB}{255,122,0}
\definecolor{tecbluelight}{RGB}{230,240,250}
\definecolor{tecorangelight}{RGB}{255,245,230}

% Enhanced color commands for consistent styling
\newcommand{\tecbluetext}[1]{\textcolor{tecblue}{\textbf{#1}}}
\newcommand{\tecorangetext}[1]{\textcolor{tecorange}{\textbf{#1}}}
\newcommand{\sectioncolor}[1]{\textcolor{tecblue}{#1}}

% Section formatting with brand colors (but tufte-book may override these)
\AtBeginDocument{
  \titleformat{\section}{\Large\bfseries\color{tecblue}}{\thesection}{1em}{}
  \titleformat{\subsection}{\large\bfseries\color{tecorange}}{\thesubsection}{1em}{}
}

% Modern colored environments using tcolorbox for better styling
\newtcolorbox{pedagogicalnote}[1][]{
  colback=tecbluelight,
  colframe=tecblue,
  coltitle=white,
  colbacktitle=tecblue,
  fonttitle=\bfseries,
  title=Pedagogical Note,
  rounded corners,
  boxrule=1pt,
  #1
}

\newtcolorbox{learningtip}[1][]{
  colback=tecorangelight,
  colframe=tecorange,
  coltitle=white,
  colbacktitle=tecorange,
  fonttitle=\bfseries,
  title=Learning Tip,
  rounded corners,
  boxrule=1pt,
  #1
}

\newtcolorbox{commonissue}[1][]{
  colback=tecorangelight,
  colframe=tecorange,
  coltitle=white,
  colbacktitle=tecorange,
  fonttitle=\bfseries,
  title=Common Issue,
  rounded corners,
  boxrule=1pt,
  #1
}

\title{Objectives of Factor Analysis}
\author{Dr. Juliho Castillo}
\date{\today}

\begin{document}
% Enhanced title block with brand colors and better typography
\begingroup
  \centering
  {\Huge\tecbluetext{Objectives of Factor Analysis}}\\[0.8em]
  {\Large\tecorangetext{Topic 4.1 - Multivariate Statistical Analysis}}\\[1.2em]
  {\large Dr. Juliho Castillo}\\[0.3em]
  {\normalsize\textsc{Tecnológico de Monterrey}}\\[0.8em]
  {\small\textcolor{tecblue}{\today}}
\par\vspace{2em}\endgroup

% Enhanced preface with color accents
\noindent\tecbluetext{About These Notes:} These notes provide a \textit{self-contained} introduction to the objectives and core concepts of factor analysis. They are designed as standalone reading material that complements but does not require the accompanying lecture slides.

\marginnote{\small\textcolor{tecorange}{\textbf{Quick Reference:} Factor analysis identifies latent variables that explain observed correlations among measured variables.}}

\tableofcontents
\newpage

\section{Lesson Overview}

\subsection{Learning Objectives}
By the end of this document the reader should be able to:
\begin{enumerate}[leftmargin=*,itemsep=0.5em]
  \item \tecbluetext{Distinguish} between factor analysis and PCA objectives
  \item \tecbluetext{Identify} when to use factor analysis versus other dimensionality reduction techniques
  \item \tecbluetext{Interpret} factor loadings and communalities in context
  \item \tecbluetext{Apply} factor analysis to real-world problems (psychology, marketing, etc.)
  \item \tecbluetext{Explain} the theoretical foundations behind factor analysis
\end{enumerate}

\marginnote{\small\textcolor{tecorange}{\textbf{Key Insight:} Factor analysis \textit{explains} correlations; PCA \textit{summarizes} variance.}}

\subsection{Prerequisites}
\begin{itemize}
  \item Basic understanding of correlation and covariance
  \item Familiarity with matrix operations
  \item Introduction to PCA
  \item Basic Python programming (for practice scripts)
\end{itemize}

\subsection{Structure}
This document contains conceptual explanations, short mathematical notes, practical advice, and suggested exercises for independent study.

\section{Introduction}

\begin{pedagogicalnote}
Start with a motivating question: ``Why do we need another dimensionality reduction technique when we already have PCA?'' 

Factor analysis addresses fundamentally different goals than PCA. While PCA finds linear combinations that capture maximum variance, factor analysis seeks to \textit{explain} the correlations between variables through underlying latent constructs.
\end{pedagogicalnote}

Factor analysis aims to uncover \textbf{latent variables} (factors) that explain observed correlations among measured variables. Unlike PCA, which finds linear combinations that summarize variance, factor analysis explicitly models measurement error and focuses on the \textit{common} variance shared among variables.

\begin{learningtip}
Remember the key distinction: \tecbluetext{PCA summarizes}; \tecorangetext{Factor Analysis explains}.
\end{learningtip}

\section{Core Objectives}

\subsection{Primary Objectives}
\begin{enumerate}[leftmargin=*,itemsep=0.8em]
  \item \tecbluetext{Dimensionality reduction} while explaining common variance
  \item \tecbluetext{Latent variable identification} — discovering constructs that cause correlations
  \item \tecbluetext{Data structure understanding} — grouping related variables conceptually
\end{enumerate}

\marginnote{\small\textcolor{tecblue}{\textbf{Historical Note:} Factor analysis was developed by psychologists like Charles Spearman (1904) to understand intelligence testing.}}

\subsection{Secondary Objectives}
\begin{enumerate}[leftmargin=*,itemsep=0.8em]
  \item \tecorangetext{Measurement purification} — separating common signal from unique measurement error
  \item \tecorangetext{Theory development} — informing construct validity and variable selection
  \item \tecorangetext{Multicollinearity reduction} — preprocessing for downstream statistical models
\end{enumerate}

\section{Factor Analysis vs PCA}

\subsection{Mathematical Contrast}
The fundamental difference lies in their mathematical formulations:

\begin{align*}
  \textcolor{tecblue}{\text{Factor model: }} &\textcolor{tecblue}{\mathbf{X} = \mathbf{\Lambda F} + \mathbf{\epsilon}}\\[0.5em]
  \textcolor{tecorange}{\text{PCA model: }} &\textcolor{tecorange}{\mathbf{X} = \mathbf{PY}}
\end{align*}

\marginnote{\small\textcolor{tecblue}{\textbf{Notation:} $\mathbf{\Lambda}$ = factor loadings, $\mathbf{F}$ = factors, $\mathbf{\epsilon}$ = unique errors}}

\subsection{Key Philosophical Differences}
\begin{enumerate}[leftmargin=*,itemsep=0.8em]
  \item \tecbluetext{Error modeling:} FA models common and unique variance explicitly; PCA does not separate measurement error
  \item \tecbluetext{Interpretability:} FA factors represent latent constructs with theoretical meaning; PCA components are mathematical summaries
  \item \tecbluetext{Parsimony:} FA typically extracts fewer factors, focusing on \textit{common} variance structure
\end{enumerate}

\begin{commonissue}
\textbf{Misconception Alert:} Negative factor loadings are \textit{meaningful} and indicate an inverse relationship between the variable and the underlying factor. Don't avoid or fear negative loadings—they provide important theoretical insights!
\end{commonissue}

\section{Applications}
Factor analysis is widely used in psychology (intelligence, personality), marketing (brand perception, segmentation), finance (risk factor identification), education, health sciences, and environmental studies.

\section{Key Concepts}

\textbf{Factor:} an unobserved variable influencing multiple observed variables.

\textbf{Loading:} correlation between observed variable and factor.

\textbf{Communality:} proportion of a variable's variance explained by the common factors.

\textbf{Uniqueness:} variance unique to the variable (including measurement error).

Simple relationship when variables are standardized: $h_i^2 + u_i^2 = 1.0$.

\section{The Factor Model}
Basic model and assumptions:
\begin{itemize}
  \item $\mathbf{X} = \mathbf{\Lambda F} + \mathbf{\epsilon}$
  \item $E[\mathbf{F}] = 0$, $E[\mathbf{\epsilon}] = 0$
  \item $\mathrm{Cov}(\mathbf{F},\mathbf{\epsilon}) = 0$
\end{itemize}

Discuss identification issues and the role of rotations (orthogonal and oblique).

\section{Types of Factor Analysis}
\textbf{Exploratory Factor Analysis (EFA):} data-driven, used for discovery.
\newline
\textbf{Confirmatory Factor Analysis (CFA):} hypothesis-driven, used to test pre-specified models.

\section{Practical Guide}

\subsection{Research problem example}
A psychologist administers six tests (Math, Logic, Spatial, Reading, Writing, Vocabulary) to 200 students and wants to know how many latent cognitive factors best explain performance.

\subsection{Analysis roadmap}
\begin{enumerate}
  \item Inspect correlations and data quality
  \item Choose extraction method and number of factors (eigenvalues, parallel analysis, scree plot)
  \item Rotate factors for interpretability
  \item Interpret loadings, communalities, and uniquenesses
  \item Validate construct using theory and external criteria
\end{enumerate}

\subsection{Interpretation checklist}
\begin{enumerate}
  \item Do the correlations make theoretical sense?
  \item Are loadings sufficiently large (commonly > 0.4--0.5)?
  \item Do communalities indicate variables are well-explained?
  \item Can factors be meaningfully named?
  \item Are results robust to different extraction/rotation choices?
\end{enumerate}

\section{Implementation notes}
The repository includes a practice script \url{objectives_factor_analysis_practice.py} which:
\begin{itemize}
  \item Generates example data
  \item Runs correlation analysis, PCA, and Factor Analysis
  \item Writes a short report file with results
\end{itemize}

To run locally from this folder:
\begin{verbatim}
python objectives_factor_analysis_practice.py
\end{verbatim}

\section{Exercises and Assessment}
\subsection{Formative questions}
\begin{enumerate}
  \item When is FA preferred to PCA? Give a real example.
  \item What does a communality of 0.7 mean in practice?
  \item How would you proceed if two variables cross-load strongly on two factors?
\end{enumerate}

\subsection{Suggested homework}
\begin{itemize}
  \item Run the practice script and write a two-page interpretation
  \item Find a published paper using FA and summarise their interpretation
  \item Propose a small study where FA would be the appropriate method
\end{itemize}

\section{Further Reading}
\begin{thebibliography}{9}
\bibitem{Tabachnick2007}
Tabachnick, B.~G. and Fidell, L.~S. (2007).
Using Multivariate Statistics.

\bibitem{Field2018}
Field, A. (2018).
Discovering Statistics with SPSS.

\bibitem{Thompson2004}
Thompson, B. (2004).
Exploratory and Confirmatory Factor Analysis.
\end{thebibliography}

\section{Online resources}
\begin{itemize}
  \item UCLA Statistical Consulting: Factor Analysis Examples
  \item StatQuest: Factor Analysis vs PCA (YouTube)
\end{itemize}

% no closing section: this is a continuous reading document

\end{document}