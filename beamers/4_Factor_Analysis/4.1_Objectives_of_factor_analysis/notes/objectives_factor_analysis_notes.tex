% Clean, self-contained Tufte-style notes for Topic 4.1: Objectives of Factor Analysis
\documentclass[a4paper,11pt]{tufte-book}
\usepackage[utf8]{inputenc}
\usepackage{amsmath,amssymb,amsfonts}
\usepackage{graphicx}
\usepackage{hyperref}
\usepackage{enumitem}
\usepackage{xcolor}
\usepackage{framed}
\usepackage{listings}

% Custom environments (boxed notes)
\definecolor{notecolor}{RGB}{255,248,220}
\definecolor{tipcolor}{RGB}{240,248,255}
\definecolor{warningcolor}{RGB}{255,240,240}

\newenvironment{pedagogicalnote}{%
  \begin{framed}
  \noindent\textbf{Pedagogical Note:}\\
}{%
  \end{framed}
}

\newenvironment{learningtip}{%
  \begin{framed}
  \noindent\textcolor{blue}{\textbf{Learning Tip:}}\\
}{%
  \end{framed}
}

\newenvironment{commonissue}{%
  \begin{framed}
  \noindent\textcolor{red}{\textbf{Common Issue:}}\\
}{%
  \end{framed}
}

\title{Objectives of Factor Analysis\\\large Topic 4.1}
\author{Dr. Juliho Castillo\\Tec de Monterrey}
\date{\today}

\begin{document}
\maketitle

% Preface: self-contained notes independent of slides
These notes are a self-contained reading resource on the objectives and core ideas of factor analysis. They are written to be understandable without the accompanying slide deck.

\tableofcontents
\newpage

\section{Lesson Overview}

\subsection{Learning Objectives}
By the end of this document the reader should be able to:
\begin{itemize}
  \item Distinguish between factor analysis and PCA objectives
  \item Identify when to use factor analysis versus other dimensionality reduction techniques
  \item Interpret factor loadings and communalities in context
  \item Apply factor analysis to real-world problems (psychology, marketing, etc.)
  \item Explain the theoretical foundations behind factor analysis
\end{itemize}

\subsection{Prerequisites}
\begin{itemize}
  \item Basic understanding of correlation and covariance
  \item Familiarity with matrix operations
  \item Introduction to PCA
  \item Basic Python programming (for practice scripts)
\end{itemize}

\subsection{Structure}
This document contains conceptual explanations, short mathematical notes, practical advice, and suggested exercises for independent study.

\section{Introduction}

\begin{pedagogicalnote}
Start with a motivating question: ``Why do we need another dimensionality reduction technique when we already have PCA?'' Factor analysis addresses different goals than PCA.
\end{pedagogicalnote}

Factor analysis aims to uncover latent variables (factors) that explain observed correlations among measured variables. PCA finds linear combinations that summarize variance but does not model measurement error explicitly.

\begin{learningtip}
Think: PCA summarizes; Factor Analysis explains.
\end{learningtip}

\section{Core Objectives}

\subsection{Primary Objectives}
\begin{itemize}
  \item Dimensionality reduction while explaining common variance
  \item Latent variable identification (discovering constructs that cause correlations)
  \item Understanding data structure and grouping related variables
\end{itemize}

\subsection{Secondary Objectives}
\begin{itemize}
  \item Measurement purification: separating common signal from unique measurement error
  \item Informing theory development and variable selection
  \item Reducing multicollinearity for downstream models
\end{itemize}

\section{Factor Analysis vs PCA}

\subsection{Mathematical contrast}
\begin{align*}
  &\text{Factor model: } \mathbf{X} = \mathbf{\Lambda F} + \mathbf{\epsilon}\\
  &\text{PCA model: } \mathbf{X} = \mathbf{PY}
\end{align*}

Key differences:
\begin{itemize}
  \item FA models common and unique variance explicitly; PCA does not
  \item FA factors are latent constructs with theoretical interpretation; PCA components are mathematical summaries
  \item FA typically extracts fewer factors and focuses on common variance
\end{itemize}

\begin{commonissue}
Negative loadings are meaningful: they indicate inverse relationships between a variable and a factor.
\end{commonissue}

\section{Applications}
Factor analysis is widely used in psychology (intelligence, personality), marketing (brand perception, segmentation), finance (risk factor identification), education, health sciences, and environmental studies.

\section{Key Concepts}

\textbf{Factor:} an unobserved variable influencing multiple observed variables.

\textbf{Loading:} correlation between observed variable and factor.

\textbf{Communality:} proportion of a variable's variance explained by the common factors.

\textbf{Uniqueness:} variance unique to the variable (including measurement error).

Simple relationship when variables are standardized: $h_i^2 + u_i^2 = 1.0$.

\section{The Factor Model}
Basic model and assumptions:
\begin{itemize}
  \item $\mathbf{X} = \mathbf{\Lambda F} + \mathbf{\epsilon}$
  \item $E[\mathbf{F}] = 0$, $E[\mathbf{\epsilon}] = 0$
  \item $\mathrm{Cov}(\mathbf{F},\mathbf{\epsilon}) = 0$
\end{itemize}

Discuss identification issues and the role of rotations (orthogonal and oblique).

\section{Types of Factor Analysis}
\textbf{Exploratory Factor Analysis (EFA):} data-driven, used for discovery.
\newline
\textbf{Confirmatory Factor Analysis (CFA):} hypothesis-driven, used to test pre-specified models.

\section{Practical Guide}

\subsection{Research problem example}
A psychologist administers six tests (Math, Logic, Spatial, Reading, Writing, Vocabulary) to 200 students and wants to know how many latent cognitive factors best explain performance.

\subsection{Analysis roadmap}
\begin{enumerate}
  \item Inspect correlations and data quality
  \item Choose extraction method and number of factors (eigenvalues, parallel analysis, scree plot)
  \item Rotate factors for interpretability
  \item Interpret loadings, communalities, and uniquenesses
  \item Validate construct using theory and external criteria
\end{enumerate}

\subsection{Interpretation checklist}
\begin{enumerate}
  \item Do the correlations make theoretical sense?
  \item Are loadings sufficiently large (commonly > 0.4--0.5)?
  \item Do communalities indicate variables are well-explained?
  \item Can factors be meaningfully named?
  \item Are results robust to different extraction/rotation choices?
\end{enumerate}

\section{Implementation notes}
The repository includes a practice script \verb|objectives_factor_analysis_practice.py| which:
\begin{itemize}
  \item Generates example data
  \item Runs correlation analysis, PCA, and Factor Analysis
  \item Writes a short report file with results
\end{itemize}

To run locally from this folder:
\begin{verbatim}
python objectives_factor_analysis_practice.py
\end{verbatim}

\section{Exercises and Assessment}
\subsection{Formative questions}
\begin{enumerate}
  \item When is FA preferred to PCA? Give a real example.
  \item What does a communality of 0.7 mean in practice?
  \item How would you proceed if two variables cross-load strongly on two factors?
\end{enumerate}

\subsection{Suggested homework}
\begin{itemize}
  \item Run the practice script and write a two-page interpretation
  \item Find a published paper using FA and summarise their interpretation
  \item Propose a small study where FA would be the appropriate method
\end{itemize}

\section{Further Reading}
\begin{itemize}
  \item Tabachnick \& Fidell: "Using Multivariate Statistics"
  \item Field: "Discovering Statistics with SPSS"
  \item Thompson: "Exploratory and Confirmatory Factor Analysis"
\end{itemize}

\section{Online resources}
\begin{itemize}
  \item UCLA Statistical Consulting: Factor Analysis Examples
  \item StatQuest: Factor Analysis vs PCA (YouTube)
\end{itemize}

% no closing section: this is a continuous reading document

\end{document}