\documentclass[aspectratio=169]{beamer}
\usetheme{metropolis}
\usecolortheme{default}
\usefonttheme{default}
\setbeamertemplate{navigation symbols}{}
\setbeamertemplate{caption}[numbered]

\usepackage[utf8]{inputenc}
\usepackage{amsmath}
\usepackage{amsfonts}
\usepackage{amssymb}
\usepackage{graphicx}
\usepackage{verbatim}

\title[Objectives of Factor Analysis]{MA2003B - Application of Multivariate Methods in Data Science}
\subtitle{Topic 4.1: Objectives of Factor Analysis}
\author{Dr. Juliho Castillo}
\institute{Tec de Monterrey}
\date{\today}

\begin{document}

\begin{frame}
  \titlepage
\end{frame}

\begin{frame}{Outline}
  \tableofcontents
\end{frame}

\section{Introduction}
\begin{frame}{What is Factor Analysis?}
  \begin{itemize}
    \item A statistical method for identifying latent variables that explain observed correlations
    \item Reduces dimensionality by finding fewer underlying factors
    \item \textbf{Goal:} Explain maximum variance with minimum number of factors
    \item Widely used in psychology, social sciences, and marketing research
  \end{itemize}
\end{frame}

\begin{frame}{Why Factor Analysis?}
  \begin{itemize}
    \item \textbf{Data Simplification:} Reduce many variables to fewer meaningful factors
    \item \textbf{Theory Testing:} Validate theoretical constructs with empirical data
    \item \textbf{Scale Development:} Create reliable measurement instruments
    \item \textbf{Noise Reduction:} Separate signal from measurement error
  \end{itemize}
\end{frame}

\section{Core Objectives}
\begin{frame}{Primary Objectives of Factor Analysis}
  \begin{enumerate}
    \item \textbf{Dimensionality Reduction}
    \begin{itemize}
      \item Transform $p$ observed variables into $k$ factors $(k < p)$
      \item Retain maximum information with fewer dimensions
    \end{itemize}
    
    \item \textbf{Latent Variable Identification}
    \begin{itemize}
      \item Discover unobserved constructs that influence multiple variables
      \item Examples: intelligence, personality traits, attitudes
    \end{itemize}
    
    \item \textbf{Data Structure Understanding}
    \begin{itemize}
      \item Identify patterns of relationships among variables
      \item Group related variables into meaningful clusters
    \end{itemize}
  \end{enumerate}
\end{frame}

\begin{frame}{Secondary Objectives}
  \begin{itemize}
    \item \textbf{Measurement Purification:} Remove measurement error from analysis
    \item \textbf{Theory Development:} Generate hypotheses about underlying structures
    \item \textbf{Variable Selection:} Identify most important variables for further analysis
    \item \textbf{Multicollinearity Reduction:} Address highly correlated predictors
  \end{itemize}
\end{frame}

\section{Factor Analysis vs PCA}
\begin{frame}{Factor Analysis vs Principal Component Analysis}
  \begin{center}
  \begin{tabular}{|l|c|c|}
    \hline
    \textbf{Aspect} & \textbf{Factor Analysis} & \textbf{PCA} \\
    \hline
    \textbf{Purpose} & Identify latent factors & Reduce dimensionality \\
    \hline
    \textbf{Model} & $\mathbf{X} = \mathbf{\Lambda F} + \mathbf{\epsilon}$ & $\mathbf{X} = \mathbf{PY}$ \\
    \hline
    \textbf{Variance Explained} & Common variance only & Total variance \\
    \hline
    \textbf{Factors/Components} & Usually $< p$ variables & Can be $= p$ variables \\
    \hline
    \textbf{Interpretation} & Latent constructs & Linear combinations \\
    \hline
    \textbf{Unique Variance} & Explicitly modeled & Not considered \\
    \hline
  \end{tabular}
  \end{center}
\end{frame}

\begin{frame}{When to Choose Factor Analysis}
  \textbf{Choose Factor Analysis when:}
  \begin{itemize}
    \item You believe latent constructs cause observed correlations
    \item Theory suggests underlying factors exist
    \item You want to model measurement error explicitly
    \item Focus is on explaining common variance only
  \end{itemize}
  
  \textbf{Choose PCA when:}
  \begin{itemize}
    \item Primary goal is dimensionality reduction
    \item You want to retain maximum total variance
    \item No specific theory about underlying factors
  \end{itemize}
\end{frame}

\section{Applications}
\begin{frame}{Applications in Psychology}
  \begin{itemize}
    \item \textbf{Intelligence Research:} Identify general intelligence factor ($g$)
    \item \textbf{Personality Assessment:} Big Five personality traits
    \item \textbf{Attitude Measurement:} Customer satisfaction, job satisfaction
    \item \textbf{Clinical Assessment:} Depression, anxiety factor structures
  \end{itemize}
\end{frame}

\begin{frame}{Applications in Business \& Marketing}
  \begin{itemize}
    \item \textbf{Customer Segmentation:} Identify customer types based on behaviors
    \item \textbf{Brand Positioning:} Understand brand perception dimensions
    \item \textbf{Product Development:} Identify key product attributes
    \item \textbf{Market Research:} Simplify complex survey data
  \end{itemize}
\end{frame}

\begin{frame}{Applications in Other Fields}
  \begin{itemize}
    \item \textbf{Finance:} Risk factors in portfolio analysis
    \item \textbf{Education:} Academic ability factors
    \item \textbf{Health Sciences:} Quality of life dimensions
    \item \textbf{Environmental Science:} Pollution factor identification
  \end{itemize}
\end{frame}

\section{Key Concepts}
\begin{frame}{Fundamental Concepts}
  \begin{itemize}
    \item \textbf{Factor:} Unobserved variable that influences multiple observed variables
    \item \textbf{Loading:} Correlation between observed variable and factor
    \item \textbf{Communality:} Proportion of variable's variance explained by all factors
    \item \textbf{Uniqueness:} Proportion of variance unique to each variable
    \item \textbf{Eigenvalue:} Amount of variance explained by each factor
  \end{itemize}
\end{frame}

\begin{frame}{The Factor Model}
  Basic factor model equation:
  $$\mathbf{X} = \mathbf{\Lambda F} + \mathbf{\epsilon}$$
  
  Where:
  \begin{itemize}
    \item $\mathbf{X}$: Observed variables $(p \times 1)$
    \item $\mathbf{\Lambda}$: Factor loadings matrix $(p \times k)$
    \item $\mathbf{F}$: Common factors $(k \times 1)$
    \item $\mathbf{\epsilon}$: Unique factors $(p \times 1)$
  \end{itemize}
  
  Key assumptions:
  \begin{itemize}
    \item $E[\mathbf{F}] = \mathbf{0}$, $E[\mathbf{\epsilon}] = \mathbf{0}$
    \item $Cov[\mathbf{F}, \mathbf{\epsilon}] = \mathbf{0}$
  \end{itemize}
\end{frame}

\section{Types of Factor Analysis}
\begin{frame}{Exploratory vs Confirmatory Factor Analysis}
  
  \textbf{Exploratory Factor Analysis (EFA):}
  \begin{itemize}
    \item No prior theory about factor structure
    \item Let data determine number of factors and loadings
    \item Used for theory generation and scale development
  \end{itemize}
  
  \textbf{Confirmatory Factor Analysis (CFA):}
  \begin{itemize}
    \item Test specific theoretical model
    \item Researcher specifies factor structure a priori
    \item Used for theory testing and model validation
  \end{itemize}
\end{frame}

\section{Hands-On Practice}

\begin{frame}{The Research Problem}
  \textbf{Scenario:} You are a psychologist studying cognitive abilities
  
  \begin{itemize}
    \item Administered 6 different tests to 200 students
    \pause
    \item \textbf{Tests:} Math, Logic, Spatial, Reading, Writing, Vocabulary
    \pause
    \item \textbf{Challenge:} Are these really 6 separate abilities?
    \pause
    \item \textbf{Hypothesis:} Maybe there are fewer underlying cognitive factors
  \end{itemize}
  
  \pause
  
  \begin{center}
    \textcolor{red}{\textbf{Research Question:} How many latent factors explain performance?}
  \end{center}
\end{frame}

\begin{frame}{Why This Matters}
  \textbf{Real-world implications:}
  
  \begin{itemize}
    \item \textbf{Educational Assessment:} Design better cognitive tests
    \pause
    \item \textbf{Personnel Selection:} Which abilities predict job performance?
    \pause
    \item \textbf{Clinical Diagnosis:} Identify specific learning difficulties
    \pause
    \item \textbf{Theory Building:} Understand the structure of human intelligence
  \end{itemize}
  
  \pause
  
  \begin{center}
    \textbf{Factor analysis helps us see the ``big picture'' behind complex data}
  \end{center}
\end{frame}

\begin{frame}{Analysis Roadmap}
  \textbf{Step-by-step approach to solve the problem:}
  
  \begin{enumerate}
    \item \textbf{Explore:} Examine correlations between test scores
    \pause
    \item \textbf{Compare Methods:} Run both PCA and Factor Analysis
    \pause
    \item \textbf{Interpret Results:} What do the factors represent?
    \pause
    \item \textbf{Validate:} Do the results make theoretical sense?
    \pause
    \item \textbf{Apply:} What would you recommend to educators?
  \end{enumerate}
  
  \pause
  
  \begin{center}
    \textcolor{blue}{\textbf{This is how real data analysis works!}}
  \end{center}
\end{frame}

\begin{frame}{Expected Outcomes}
  \textbf{What should we expect to find?}
  
  \begin{itemize}
    \item \textbf{High correlations} within groups (Math-Logic-Spatial, Reading-Writing-Vocabulary)
    \pause
    \item \textbf{Two main factors} emerging from the analysis
    \pause
    \item \textbf{Factor 1:} ``Intelligence'' (quantitative/spatial reasoning)
    \pause
    \item \textbf{Factor 2:} ``Verbal Ability'' (language-based skills)
  \end{itemize}
  
  \pause
  
  \begin{center}
    \textbf{Will our analysis confirm this psychological theory?}
  \end{center}
\end{frame}

\section{Running the Analysis}

\begin{frame}{How to Use the Practice Script}
  \textbf{Simple command to run:}
  
  \begin{center}
    \Large \texttt{python objectives\_factor\_analysis\_practice.py}
  \end{center}
  
  \pause
  
  \textbf{What it does:}
  \begin{itemize}
    \item Generates the psychology test data
    \pause
    \item Runs correlation analysis, PCA, and Factor Analysis
    \pause
    \item Saves complete results to \texttt{factor\_analysis\_report.txt}
  \end{itemize}
  
  \pause
  
  \begin{center}
    \textcolor{green}{\textbf{Everything is automated - just run and analyze results!}}
  \end{center}
\end{frame}

\begin{frame}{How to Interpret Your Results}
  \textbf{Look for these key findings in your report:}
  
  \begin{enumerate}
    \item \textbf{Correlation Matrix:} Which test scores are highly correlated?
    \pause
    \item \textbf{Factor Loadings:} Which tests load on which factors?
    \pause
    \item \textbf{Communalities:} How much of each test is explained by factors?
    \pause
    \item \textbf{Factor Interpretation:} Do the factors match our theory?
  \end{enumerate}
  
  \pause
  
  \begin{center}
    \textbf{Your job: Make sense of the statistical output!}
  \end{center}
\end{frame}

\begin{frame}{Discussion Questions}
  \textbf{After reviewing your results, consider:}
  
  \begin{itemize}
    \item Did we find the expected 2-factor structure?
    \pause
    \item Do Math, Logic, Spatial tests load on one factor?
    \pause
    \item Do Reading, Writing, Vocabulary tests load on another factor?
    \pause
    \item How would you explain these factors to a school principal?
    \pause
    \item What practical recommendations would you make?
  \end{itemize}
  
  \pause
  
  \begin{center}
    \textcolor{red}{\textbf{This is the real skill: Turning statistics into insights!}}
  \end{center}
\end{frame}

\begin{frame}
  \centering
  \Huge Thank You!
  
  \normalsize Questions?
\end{frame}

\end{document}