\documentclass[aspectratio=169]{beamer}
\usetheme{metropolis}
\usecolortheme{default}
\usefonttheme{default}
\setbeamertemplate{navigation symbols}{}
\setbeamertemplate{caption}[numbered]

\usepackage[utf8]{inputenc}
\usepackage{amsmath}
\usepackage{amsfonts}
\usepackage{amssymb}
\usepackage{graphicx}
\usepackage{verbatim}

\title[Objectives of Factor Analysis]{MA2003B - Application of Multivariate Methods in Data Science}
\subtitle{Topic 4.1: Objectives of Factor Analysis}
\author{Dr. Juliho Castillo}
\institute{Tec de Monterrey}
\date{\today}

\begin{document}

\begin{frame}
  \titlepage
\end{frame}

\begin{frame}{Outline}
  \tableofcontents
\end{frame}

\section{Introduction}
\begin{frame}{What is Factor Analysis?}
  \begin{itemize}
    \item A statistical method for identifying latent variables that explain observed correlations
    \item Reduces dimensionality by finding fewer underlying factors
    \item \textbf{Goal:} Explain maximum variance with minimum number of factors
    \item Widely used in psychology, social sciences, and marketing research
  \end{itemize}
\end{frame}

\begin{frame}{Why Factor Analysis?}
  \begin{itemize}
    \item \textbf{Data Simplification:} Reduce many variables to fewer meaningful factors
    \item \textbf{Theory Testing:} Validate theoretical constructs with empirical data
    \item \textbf{Scale Development:} Create reliable measurement instruments
    \item \textbf{Noise Reduction:} Separate signal from measurement error
  \end{itemize}
\end{frame}

\section{Core Objectives}
\begin{frame}{Primary Objectives of Factor Analysis}
  \begin{enumerate}
    \item \textbf{Dimensionality Reduction}
    \begin{itemize}
      \item Transform $p$ observed variables into $k$ factors $(k < p)$
      \item Retain maximum information with fewer dimensions
    \end{itemize}
    
    \item \textbf{Latent Variable Identification}
    \begin{itemize}
      \item Discover unobserved constructs that influence multiple variables
      \item Examples: intelligence, personality traits, attitudes
    \end{itemize}
    
    \item \textbf{Data Structure Understanding}
    \begin{itemize}
      \item Identify patterns of relationships among variables
      \item Group related variables into meaningful clusters
    \end{itemize}
  \end{enumerate}
\end{frame}

\begin{frame}{Secondary Objectives}
  \begin{itemize}
    \item \textbf{Measurement Purification:} Remove measurement error from analysis
    \item \textbf{Theory Development:} Generate hypotheses about underlying structures
    \item \textbf{Variable Selection:} Identify most important variables for further analysis
    \item \textbf{Multicollinearity Reduction:} Address highly correlated predictors
  \end{itemize}
\end{frame}

\section{Factor Analysis vs PCA}
\begin{frame}{Factor Analysis vs Principal Component Analysis}
  \begin{center}
  \begin{tabular}{|l|c|c|}
    \hline
    \textbf{Aspect} & \textbf{Factor Analysis} & \textbf{PCA} \\
    \hline
    \textbf{Purpose} & Identify latent factors & Reduce dimensionality \\
    \hline
    \textbf{Model} & $\mathbf{X} = \mathbf{\Lambda F} + \mathbf{\epsilon}$ & $\mathbf{X} = \mathbf{PY}$ \\
    \hline
    \textbf{Variance Explained} & Common variance only & Total variance \\
    \hline
    \textbf{Factors/Components} & Usually $< p$ variables & Can be $= p$ variables \\
    \hline
    \textbf{Interpretation} & Latent constructs & Linear combinations \\
    \hline
    \textbf{Unique Variance} & Explicitly modeled & Not considered \\
    \hline
  \end{tabular}
  \end{center}
\end{frame}

\begin{frame}{When to Choose Factor Analysis}
  \textbf{Choose Factor Analysis when:}
  \begin{itemize}
    \item You believe latent constructs cause observed correlations
    \item Theory suggests underlying factors exist
    \item You want to model measurement error explicitly
    \item Focus is on explaining common variance only
  \end{itemize}
  
  \vspace{0.5cm}
  
  \textbf{Choose PCA when:}
  \begin{itemize}
    \item Primary goal is dimensionality reduction
    \item You want to retain maximum total variance
    \item No specific theory about underlying factors
  \end{itemize}
\end{frame}

\section{Applications}
\begin{frame}{Applications in Psychology}
  \begin{itemize}
    \item \textbf{Intelligence Research:} Identify general intelligence factor ($g$)
    \item \textbf{Personality Assessment:} Big Five personality traits
    \item \textbf{Attitude Measurement:} Customer satisfaction, job satisfaction
    \item \textbf{Clinical Assessment:} Depression, anxiety factor structures
  \end{itemize}
\end{frame}

\begin{frame}{Applications in Business \& Marketing}
  \begin{itemize}
    \item \textbf{Customer Segmentation:} Identify customer types based on behaviors
    \item \textbf{Brand Positioning:} Understand brand perception dimensions
    \item \textbf{Product Development:} Identify key product attributes
    \item \textbf{Market Research:} Simplify complex survey data
  \end{itemize}
\end{frame}

\begin{frame}{Applications in Other Fields}
  \begin{itemize}
    \item \textbf{Finance:} Risk factors in portfolio analysis
    \item \textbf{Education:} Academic ability factors
    \item \textbf{Health Sciences:} Quality of life dimensions
    \item \textbf{Environmental Science:} Pollution factor identification
  \end{itemize}
\end{frame}

\section{Key Concepts}
\begin{frame}{Fundamental Concepts}
  \begin{itemize}
    \item \textbf{Factor:} Unobserved variable that influences multiple observed variables
    \item \textbf{Loading:} Correlation between observed variable and factor
    \item \textbf{Communality:} Proportion of variable's variance explained by all factors
    \item \textbf{Uniqueness:} Proportion of variance unique to each variable
    \item \textbf{Eigenvalue:} Amount of variance explained by each factor
  \end{itemize}
\end{frame}

\begin{frame}{The Factor Model}
  Basic factor model equation:
  $$\mathbf{X} = \mathbf{\Lambda F} + \mathbf{\epsilon}$$
  
  Where:
  \begin{itemize}
    \item $\mathbf{X}$: Observed variables $(p \times 1)$
    \item $\mathbf{\Lambda}$: Factor loadings matrix $(p \times k)$
    \item $\mathbf{F}$: Common factors $(k \times 1)$
    \item $\mathbf{\epsilon}$: Unique factors $(p \times 1)$
  \end{itemize}
  
  \vspace{0.3cm}
  Key assumptions:
  \begin{itemize}
    \item $E[\mathbf{F}] = \mathbf{0}$, $E[\mathbf{\epsilon}] = \mathbf{0}$
    \item $Cov[\mathbf{F}, \mathbf{\epsilon}] = \mathbf{0}$
  \end{itemize}
\end{frame}

\section{Types of Factor Analysis}
\begin{frame}{Exploratory vs Confirmatory Factor Analysis}
  
  \textbf{Exploratory Factor Analysis (EFA):}
  \begin{itemize}
    \item No prior theory about factor structure
    \item Let data determine number of factors and loadings
    \item Used for theory generation and scale development
  \end{itemize}
  
  \vspace{0.5cm}
  
  \textbf{Confirmatory Factor Analysis (CFA):}
  \begin{itemize}
    \item Test specific theoretical model
    \item Researcher specifies factor structure a priori
    \item Used for theory testing and model validation
  \end{itemize}
\end{frame}

\section{Python Practice}
\begin{frame}{Python Practice: Factor Analysis Objectives}
  \begin{itemize}
    \item Try the script: \texttt{objectives\_factor\_analysis\_practice.py}
    \item Demonstrates factor analysis vs PCA comparison
    \item Shows practical applications with real data examples
    \item Illustrates key concepts and interpretations
  \end{itemize}
\end{frame}

\begin{frame}[fragile]{Running the Script}
  \begin{itemize}
    \item Install required libraries if needed:
  \end{itemize}
  \vspace{0.5cm}
  \begin{verbatim}
pip install numpy scipy scikit-learn factor-analyzer
  \end{verbatim}
  \vspace{0.5cm}
  \begin{itemize}
    \item Run the script:
  \end{itemize}
  \vspace{0.5cm}
  \begin{verbatim}
python objectives_factor_analysis_practice.py
  \end{verbatim}
\end{frame}

\begin{frame}
  \centering
  \Huge Thank You!
  \vspace{1cm}
  \normalsize Questions?
\end{frame}

\end{document}