\documentclass[aspectratio=169]{beamer}
\usetheme{metropolis}
\usecolortheme{default}
\usefonttheme{default}
\setbeamertemplate{navigation symbols}{}
\setbeamertemplate{caption}[numbered]

\usepackage[utf8]{inputenc}
\usepackage{amsmath}
\usepackage{amsfonts}
\usepackage{amssymb}
\usepackage{graphicx}
\usepackage{verbatim}

\title[Simple Linear Regression]{MA2003B - Application of Multivariate Methods in Data Science}
\subtitle{Topic 1.1: Simple Linear Regression}
\author{Dr. Juliho Castillo}
\institute{Tec de Monterrey}
\date{\today}

\begin{document}

\begin{frame}
  \titlepage
\end{frame}

\begin{frame}{Outline}
  \tableofcontents
\end{frame}

\section{Introduction}
\begin{frame}{What is Simple Linear Regression?}
  \begin{itemize}
    \item A method to study the relationship between two quantitative variables.
    \item One variable ($X$): predictor/independent. One variable ($Y$): response/dependent.
    \item \textbf{Goal:} Find and describe a linear relationship between $X$ and $Y$.
    \item Used for description and prediction.
  \end{itemize}
\end{frame}

\section{The Model}
\begin{frame}{The Simple Linear Regression Model}
  \begin{itemize}
    \item The model: $Y_i = \beta_0 + \beta_1 X_i + \epsilon_i$
    \item $\beta_0$: intercept (value of $Y$ when $X=0$)
    \item $\beta_1$: slope (change in $Y$ per unit change in $X$)
    \item $\epsilon_i$: random error (unexplained variation)
  \end{itemize}
\end{frame}

\section{Assumptions}
\begin{frame}{Key Assumptions (Brief)}
  \begin{itemize}
    \item Linearity: Relationship between $X$ and $Y$ is linear.
    \item Independence: Observations are independent.
    \item Constant variance: Spread of errors is similar for all $X$.
    \item Normality: Errors are roughly normally distributed.
  \end{itemize}
\end{frame}

\section{Estimation}
\begin{frame}{Estimating the Model}
  \begin{itemize}
    \item We estimate $\beta_0$ and $\beta_1$ from data (using least squares).
    \item The fitted line: $\hat{Y}_i = b_0 + b_1 X_i$
    \item $b_0$ and $b_1$ are calculated to best fit the observed data.
    \item (Details and formulas are covered in practice/code.)
  \end{itemize}
\end{frame}

\section{Interpretation}
\begin{frame}{Interpreting the Coefficients}
  \begin{itemize}
    \item $b_0$: Estimated value of $Y$ when $X=0$ (may not always be meaningful).
    \item $b_1$: Estimated change in $Y$ for a one-unit increase in $X$.
    \item If $b_1 > 0$: $Y$ increases with $X$. If $b_1 < 0$: $Y$ decreases with $X$.
  \end{itemize}
\end{frame}

\section{Model Fit}
\begin{frame}{How Good is the Model? ($R^2$)}
  \begin{itemize}
    \item $R^2$ (coefficient of determination): Proportion of variability in $Y$ explained by $X$.
    \item $R^2$ ranges from 0 (no fit) to 1 (perfect fit).
    \item Higher $R^2$ means a better fit.
  \end{itemize}
\end{frame}

\section{Python Practice}
\begin{frame}{Python Practice: Simple Linear Regression}
  \begin{itemize}
    \item Try the provided script: \texttt{simple\_linear\_regression\_practice.py}
    \item It demonstrates fitting a model, interpreting coefficients, and $R^2$.
    \item Follow the comments in the script for guidance.
  \end{itemize}
\end{frame}

\begin{frame}[fragile]{Running the Script}
  \begin{itemize}
    \item Install required libraries if needed:
  \end{itemize}
  \begin{verbatim}
pip install numpy scikit-learn matplotlib
  \end{verbatim}
  \begin{itemize}
    \item Run the script:
  \end{itemize}
  \begin{verbatim}
python simple_linear_regression_practice.py
  \end{verbatim}
\end{frame}

\begin{frame}
  \centering
  \Huge Thank You!
  \vspace{1cm}
  \normalsize Questions?
\end{frame}

\end{document}
