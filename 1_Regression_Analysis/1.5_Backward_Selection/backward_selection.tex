\documentclass[aspectratio=169]{beamer}
\usetheme{metropolis}
\usecolortheme{default}
\usefonttheme{default}
\setbeamertemplate{navigation symbols}{}
\setbeamertemplate{caption}[numbered]

\usepackage[utf8]{inputenc}
\usepackage{amsmath}
\usepackage{amsfonts}
\usepackage{amssymb}
\usepackage{graphicx}
\usepackage{listings}
\usepackage{xcolor}

\lstset{
    basicstyle=\ttfamily\small,
    breaklines=true,
    frame=single,
    backgroundcolor=\color{gray!10},
    commentstyle=\color{green!50!black},
    keywordstyle=\color{blue},
    stringstyle=\color{red},
    showstringspaces=false
}

\title[Backward Selection]{MA2003B - Application of Multivariate Methods in Data Science}
\subtitle{Topic 1.5: Backward Selection}
\author{Dr. Juliho Castillo}
\institute{Tec de Monterrey}
\date{\today}

\begin{document}

\begin{frame}
  \titlepage
\end{frame}

\begin{frame}{Outline}
  \tableofcontents
\end{frame}

\section{Introduction}
\begin{frame}{What is Backward Selection?}
  \begin{itemize}
    \item Backward selection is a stepwise regression method.
    \item It starts with all predictors and removes them one by one.
    \item At each step, the predictor that contributes the least is removed.
    \item Useful for simplifying models.
  \end{itemize}
\end{frame}

\section{Steps in Backward Selection}
\begin{frame}{How Does Backward Selection Work?}
  \begin{itemize}
    \item Start with a full model including all predictors.
    \item Evaluate predictors and remove the one with the least contribution (e.g., highest p-value).
    \item Repeat until all remaining predictors are significant.
    \item Stop when removing predictors no longer improves the model.
  \end{itemize}
\end{frame}

\section{Python Practice}
\begin{frame}{Python Practice: Backward Selection}
  \begin{itemize}
    \item Try the script: \texttt{backward\_selection\_practice.py}
    \item It demonstrates how to implement backward selection using statsmodels.
    \item Follow the comments in the script for guidance.
  \end{itemize}
\end{frame}

\begin{frame}[fragile]{Running the Script}
  \begin{itemize}
    \item Install required libraries if needed:
  \end{itemize}
  \vspace{0.5cm} % Add spacing before code block
  \begin{lstlisting}[language=bash]
pip install numpy pandas statsmodels matplotlib
  \end{lstlisting}
  \vspace{0.5cm} % Add spacing after code block
  \begin{itemize}
    \item Run the script.
    \item Check the generated report for results.
  \end{itemize}
\end{frame}

\begin{frame}
  \centering
  \Huge Thank You!
  \vspace{1cm}
\end{frame}

\end{document}
