\documentclass[aspectratio=169]{beamer}

% Theme and packages
\usetheme{metropolis}
\usecolortheme{default}
\usefonttheme{default}
\setbeamertemplate{navigation symbols}{}
\setbeamertemplate{caption}[numbered]

\usepackage[utf8]{inputenc}
\usepackage{amsmath}
\usepackage{amsfonts}
\usepackage{amssymb}
\usepackage{graphicx}
\usepackage{listings}
\usepackage{xcolor}

% Configure listings
\lstset{
    basicstyle=\ttfamily\small,
    breaklines=true,
    frame=single,
    backgroundcolor=\color{gray!10},
    commentstyle=\color{green!50!black},
    keywordstyle=\color{blue},
    stringstyle=\color{red},
    showstringspaces=false
}

\title[ANOVA \& Confidence Intervals]{MA2003B - Application of Multivariate Methods in Data Science}
\subtitle{Topic 1.2: ANOVA and Confidence Intervals}
\author{Dr. Juliho Castillo}
\institute{Tec de Monterrey}
\date{\today}

\begin{document}

\begin{frame}
  \titlepage
\end{frame}

\begin{frame}{Outline}
  \tableofcontents
\end{frame}

\section{Introduction}
\begin{frame}{What is ANOVA?}
  \begin{itemize}
    \item \textbf{ANOVA (Analysis of Variance)} is a statistical method used to compare means across multiple groups.
    \item It helps determine if at least one group mean is significantly different from the others.
    \item Commonly used in experimental design and regression analysis.
  \end{itemize}
\end{frame}

\section{ANOVA in Regression}
\begin{frame}{ANOVA Table for Regression}
  \begin{itemize}
    \item ANOVA partitions the total variability in the response variable ($Y$) into components:
    \begin{itemize}
      \item \textbf{Regression (SSR):} Variability explained by the model.
      \item \textbf{Error (SSE):} Variability not explained by the model.
      \item \textbf{Total (SST):} Total variability in $Y$.
    \end{itemize}
    \item The ANOVA table summarizes these components:
  \end{itemize}
  \begin{center}
    \begin{tabular}{lccc}
      Source & Sum of Squares & df & Mean Square \\
      \hline
      Regression & SSR & 1 & MSR = SSR/1 \\
      Error & SSE & n-2 & MSE = SSE/(n-2) \\
      Total & SST & n-1 & \\
    \end{tabular}
  \end{center}
\end{frame}

\begin{frame}{F-Test in ANOVA}
  \begin{itemize}
    \item The F-test assesses whether the regression model explains a significant portion of the variability in $Y$.
    \item \textbf{Null Hypothesis ($H_0$):} $\beta_1 = 0$ (no linear relationship)
    \item \textbf{Alternative Hypothesis ($H_a$):} $\beta_1 \neq 0$
    \item \textbf{Test Statistic:} $F = \frac{MSR}{MSE}$
    \item If $F$ is large (p-value $< \alpha$), reject $H_0$.
  \end{itemize}
\end{frame}

\section{Confidence Intervals}
\begin{frame}{Confidence Intervals in Regression}
  \begin{itemize}
    \item Confidence intervals provide a range of plausible values for population parameters (e.g., slope, mean response).
    \item \textbf{For the slope ($\beta_1$):} $b_1 \pm t_{\alpha/2, n-2} \times SE(b_1)$
    \item \textbf{For the mean response at $X_h$:} $\hat{Y}_h \pm t_{\alpha/2, n-2} \times SE(\hat{Y}_h)$
    \item \textbf{For a new observation (prediction interval):} $\hat{Y}_h \pm t_{\alpha/2, n-2} \times SE(pred)$
  \end{itemize}
\end{frame}

\begin{frame}{Interpreting Confidence Intervals}
  \begin{itemize}
    \item If the confidence interval for $\beta_1$ does not contain 0, there is evidence of a significant linear relationship.
    \item Wider intervals indicate more uncertainty in the estimate.
    \item Confidence intervals for predictions are wider than for the mean response.
  \end{itemize}
\end{frame}

\section{Python Code Practice}
\begin{frame}{Python Practice: ANOVA and Confidence Intervals}
  \begin{itemize}
    \item A Python script named \texttt{anova\_confidence\_intervals\_practice.py} is provided in this directory.
    \item This script demonstrates ANOVA and confidence intervals using \texttt{numpy}, \texttt{scipy}, and \texttt{statsmodels}.
    \item Open the script and follow the instructions within the comments to:
    \begin{itemize}
      \item Fit a regression model and extract ANOVA table.
      \item Calculate and interpret confidence intervals for the slope and mean response.
      \item Make predictions with prediction intervals.
    \end{itemize}
  \end{itemize}
\end{frame}

\begin{frame}[fragile]{Python Practice: Running the Script}
  \vspace{-0.2cm}
  \begin{itemize}
    \item[\textbf{Step 1:}] Ensure you have Python installed with the necessary libraries: \texttt{numpy}, \texttt{scipy}, \texttt{statsmodels}.
    \item[\textbf{Step 2:}] If needed, install them using pip:
  \end{itemize}
  
  \begin{lstlisting}[language=bash]
pip install numpy scipy statsmodels
  \end{lstlisting}
  
  \begin{itemize}
    \item[\textbf{Step 3:}] Navigate to the directory containing the Python script.
    \item[\textbf{Step 4:}] Run the script from your terminal:
  \end{itemize}
  
  \begin{lstlisting}[language=bash]
python anova_confidence_intervals_practice.py
  \end{lstlisting}
  
  \begin{itemize}
    \item[\textbf{Step 5:}] Observe the output and complete the interpretation tasks.
  \end{itemize}
\end{frame}

\begin{frame}
  \centering
  \Huge Thank You!
  \vspace{1cm}
  \normalsize Questions?
\end{frame}

\end{document}
