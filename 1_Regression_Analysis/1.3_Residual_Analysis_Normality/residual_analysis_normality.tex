\documentclass[aspectratio=169]{beamer}
\usetheme{metropolis}
\usecolortheme{default}
\usefonttheme{default}
\setbeamertemplate{navigation symbols}{}
\setbeamertemplate{caption}[numbered]

\usepackage[utf8]{inputenc}
\usepackage{amsmath}
\usepackage{amsfonts}
\usepackage{amssymb}
\usepackage{graphicx}
\usepackage{listings}
\usepackage{xcolor}

\lstset{
    basicstyle=\ttfamily\small,
    breaklines=true,
    frame=single,
    backgroundcolor=\color{gray!10},
    commentstyle=\color{green!50!black},
    keywordstyle=\color{blue},
    stringstyle=\color{red},
    showstringspaces=false
}

\title[Residual Analysis \& Normality]{MA2003B - Application of Multivariate Methods in Data Science}
\subtitle{Topic 1.3: Residual Analysis and Normality}
\author{Dr. Juliho Castillo}
\institute{Tec de Monterrey}
\date{\today}

\begin{document}

\begin{frame}
  \titlepage
\end{frame}

\begin{frame}{Outline}
  \tableofcontents
\end{frame}

\section{Introduction}
\begin{frame}{Why Analyze Residuals?}
  \begin{itemize}
    \item Residuals are the differences between observed and predicted values: $e_i = y_i - \hat{y}_i$
    \item Analyzing residuals helps check if the regression model assumptions are met.
    \item Key for validating the quality and reliability of your model.
  \end{itemize}
\end{frame}

\section{What to Look For}
\begin{frame}{What Should Residuals Look Like?}
  \begin{itemize}
    \item Residuals should have no clear pattern (random scatter).
    \item Residuals should have constant spread (homoscedasticity).
    \item Residuals should be roughly normally distributed.
    \item Patterns or non-constant spread suggest model problems.
  \end{itemize}
\end{frame}

\section{Checking Residuals}
\begin{frame}{How to Check Residuals}
  \begin{itemize}
    \item Plot residuals vs. fitted values (look for randomness)
    \item Histogram or Q-Q plot of residuals (check normality)
    \item Look for outliers or influential points
  \end{itemize}
\end{frame}

\section{Normality}
\begin{frame}{Normality of Residuals}
  \begin{itemize}
    \item Many regression inferences assume residuals are normal.
    \item Normality can be checked visually (histogram, Q-Q plot) or with tests (e.g., Shapiro-Wilk).
    \item Mild deviations are usually OK; strong non-normality may require a different model.
  \end{itemize}
\end{frame}

\section{Python Practice}
\begin{frame}{Python Practice: Residual Analysis}
  \begin{itemize}
    \item Try the script: \texttt{residual\_analysis\_normality\_practice.py}
    \item It shows how to plot and analyze residuals and check normality.
    \item Follow the comments in the script for guidance.
  \end{itemize}
\end{frame}

\begin{frame}[fragile]{Running the Script}
  \begin{itemize}
    \item Install required libraries if needed:
  \end{itemize}
  \begin{lstlisting}[language=bash]
pip install numpy pandas statsmodels matplotlib scipy
  \end{lstlisting}
  \begin{itemize}
    \item Run the script:
  \end{itemize}
  \begin{lstlisting}[language=bash]
python residual_analysis_normality_practice.py
  \end{lstlisting}
\end{frame}

\begin{frame}
  \centering
  \Huge Thank You!
  \vspace{1cm}
  \normalsize Questions?
\end{frame}

\end{document}
