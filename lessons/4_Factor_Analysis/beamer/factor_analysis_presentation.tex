\documentclass[aspectratio=169]{beamer}

% Presentation metadata
\title{Factor Analysis}
\author{Dr. Juliho Castillo}
\institute{Tecnológico de Monterrey}
\date{\today}

% Additional packages for the presentation
\usepackage[utf8]{inputenc}
\usepackage[T1]{fontenc}
\usepackage[spanish]{babel}
\usepackage{amsmath,amssymb}
\usepackage{graphicx}
\usepackage{booktabs}
\usepackage{multirow}
\usepackage{listings}
\usepackage{xcolor}


\begin{document}

% Title slide
\begin{frame}
    \titlepage
\end{frame}

% Table of contents
\begin{frame}
    \tableofcontents
\end{frame}

% Section: Introducción

\section{Introduction}

\begin{frame}[fragile]
\frametitle{What is Factor Analysis?}
\begin{itemize}
    \item A statistical method for modeling relationships among \textbf{observed variables}. \pause
    \item It uses a smaller number of \textit{unobserved variables}, known as \textbf{factors}. \pause
    \item Often used in an \textit{unsupervised manner} to discover underlying patterns.
\end{itemize}
\end{frame}

\begin{frame}[fragile]
\frametitle{Key Concepts}
\begin{itemize}
    \item \textbf{Factors (or Latent Variables):} \pause
    \begin{itemize}
        \item These are the underlying, unobserved variables. \pause
        \item Example: A latent variable like "fairness" might be composed of observed variables like "demeanor" and "preparation for trial."
    \end{itemize}
    \item \textbf{The Core Assumption:} \pause
    \begin{itemize}
        \item Observed variables are a linear combination of a few common factors and a unique factor for each variable.
    \end{itemize}
\end{itemize}
\end{frame}

\begin{frame}[fragile]
\frametitle{Factor Analysis vs. Principal Component Analysis (PCA)}
\begin{itemize}
    \item \textbf{PCA:} Primarily a \textit{dimensionality reduction} technique. It focuses on summarizing the data by finding the directions of maximum variance. \pause
    \item \textbf{Factor Analysis:} Aims to explain the \textit{latent structure} of the data and identify the underlying constructs that explain the observed correlations. \pause
    \item \textbf{Summary:} While both reduce the number of variables, they have different goals.
\end{itemize}
\end{frame}

\begin{frame}[fragile]
\frametitle{A Word of Caution}
\begin{itemize}
    \item Factor analysis is a \textit{modeling technique}. \pause
    \item A confirmatory factor analysis cannot \textit{prove} a model is correct, only that it is plausible. \pause
    \item The method can be ``fragile'' because multiple, equally valid models might exist for the same dataset.
\end{itemize}
\end{frame}



\end{document}
