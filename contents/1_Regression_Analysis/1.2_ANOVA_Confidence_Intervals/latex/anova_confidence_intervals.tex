\documentclass[aspectratio=169]{beamer}
\usetheme{metropolis}
\usecolortheme{default}
\usefonttheme{default}
\setbeamertemplate{navigation symbols}{}
\setbeamertemplate{caption}[numbered]

\usepackage[utf8]{inputenc}
\usepackage{amsmath}
\usepackage{amsfonts}
\usepackage{amssymb}
\usepackage{graphicx}
\usepackage{verbatim}

\title[ANOVA \& Confidence Intervals]{MA2003B - Application of Multivariate Methods in Data Science}
\subtitle{Topic 1.2: ANOVA and Confidence Intervals}
\author{Dr. Juliho Castillo}
\institute{Tec de Monterrey}
\date{\today}

\begin{document}

\begin{frame}
  \titlepage
\end{frame}

\begin{frame}{Outline}
  \tableofcontents
\end{frame}

\section{Introduction}
\begin{frame}{What is ANOVA?}
  \begin{itemize}
    \item ANOVA (Analysis of Variance) compares means across groups or models.
    \item In regression, it helps test if the model explains a significant part of the variation in $Y$.
    \item Used to check if the relationship between $X$ and $Y$ is statistically significant.
  \end{itemize}
\end{frame}

\section{ANOVA Table}
\begin{frame}{ANOVA Table in Regression}
  \begin{itemize}
    \item ANOVA splits the total variation in $Y$ into:
    \begin{itemize}
      \item Variation explained by the model (regression)
      \item Unexplained variation (error)
    \end{itemize}
    \item The table summarizes these parts and their degrees of freedom.
  \end{itemize}
  \begin{center}
    \begin{tabular}{lccc}
      Source & Sum of Squares & df & Mean Square \\
      \hline
      Regression & SSR & 1 & MSR \\
      Error & SSE & n-2 & MSE \\
      Total & SST & n-1 & \\
    \end{tabular}
  \end{center}
\end{frame}

\section{F-Test}
\begin{frame}{The F-Test (Idea)}
  \begin{itemize}
    \item The F-test checks if the regression model explains a significant amount of variation in $Y$.
    \item \textbf{Null hypothesis:} No relationship between $X$ and $Y$.
    \item \textbf{Alternative:} There is a relationship.
    \item If the F-statistic is large (p-value small), the model is significant.
    \item (Details and formulas are covered in practice/code.)
  \end{itemize}
\end{frame}

\section{Confidence Intervals}
\begin{frame}{Confidence Intervals (Brief)}
  \begin{itemize}
    \item Confidence intervals give a range of plausible values for model parameters (like the slope).
    \item If the interval for the slope does not include 0, the relationship is significant.
    \item (How to compute and interpret is shown in practice/code.)
  \end{itemize}
\end{frame}

\section{Python Practice}
\begin{frame}{Python Practice: ANOVA and Confidence Intervals}
  \begin{itemize}
    \item Try the script: \texttt{anova\_confidence\_intervals\_practice.py}
    \item It shows how to fit a model, get the ANOVA table, and confidence intervals.
    \item Follow the comments in the script for guidance.
  \end{itemize}
\end{frame}

\begin{frame}[fragile]{Running the Script}
  \begin{itemize}
    \item Install required libraries if needed:
  \end{itemize}
  \begin{verbatim}
pip install numpy pandas statsmodels
  \end{verbatim}
  \begin{itemize}
    \item Run the script:
  \end{itemize}
  \begin{verbatim}
python anova_confidence_intervals_practice.py
  \end{verbatim}
\end{frame}

\begin{frame}
  \centering
  \Huge Thank You!
  \vspace{1cm}
  \normalsize Questions?
\end{frame}

\end{document}
